\documentclass[conference,twocolumn]{IEEEtran}
\usepackage[backend=bibtex,firstinits=true]{biblatex}
\addbibresource{paper}
\makeatletter\def\blx@maxline{77}\makeatother

\usepackage{graphicx}
\newcommand{\rb}[1]{\rotatebox{90}{#1}}

\usepackage{hyperref}
\usepackage{mathtools}
\usepackage{booktabs}

\title{Common Concerns in BYOD Policies}
\author{%
  \IEEEauthorblockN{Joseph Hallett}
  \IEEEauthorblockA{University of Edinburgh}
  \and
  \IEEEauthorblockN{David Aspinall}
  \IEEEauthorblockA{University of Edinburgh}
}

\begin{document}
\maketitle

\begin{abstract}
  Companies publish BYOD policies for employees to use their device at work.
  These policies are written using natural language, and vary in length and style.
  Using BYOD policies from five different sources we explore common concerns.
\end{abstract}

\section{Introduction}
\label{sec:introduction}

Employees increasingly bring their personal mobile devices to work.
To control the access these devices have, 70\% of companies publish \emph{bring your own device} (BYOD) policies~\cite{schulze_byod_2016}.
These BYOD policies are written documents for employees to read and follow.

Companies have a variety of means to implement their policies.
Some companies may trust employees to follow the rules on their own.
Alternatively \emph{Mobile Device Management} (MDM) software can implement part of the policies: packages such as IBM's MaaS360 and Blackberry's offering~\cite{_ibm_????,_secure_????} can configure devices to restrict functionality and manage apps.
Research has looked at developing other tools such as UC-Droid~\cite{martinelli_enhancing_2016} or BYODroid~\cite{armando_enabling_2014} which was used to implement parts of a NATO BYOD policy~\cite{armando_developing_2016}.

Commercial tools are limited in what polices they can enforce.
The tools are limited to simple on-off configuration settings, and banning explicitly black-listed apps.
More sophisticated systems can use app-rewriting to recompile apps to tunnel traffic through a VPN, or geo-fencing to only apply policies in predefined areas.
These tools are not infallible.
One survey found that 50\% of companies with MDM software still had non-compliant devices in their networks~\cite{mobileiron_security_labs_q4_2015}.
Whilst app wrapping can protect some apps, in general it is ineffective~\cite{hao_effectiveness_2013}.

MDM software and research has, so far, focussed on restricting apps and device functionality.
But what are the concerns and restriction described in the policies themselves?
By analyzing 5 BYOD policies, we present the common concerns and structures found in the policies themselves.
This suggests where future MDM tools should focus their efforts.

\section{BYOD Policies}
\label{sec:byod_policies}

We analyzed 5 policies from different sources to find common concerns.
The SANS policy~\cite{nicholas_r._c._guerin_security_2008} is a hypothetical BYOD policy that companies can use as a starting point to base their policy on.
It is prescriptive and long, focussed on technical restrictions.
The HiMSS policy~\cite{healthcare_information_and_management_systems_society_mobile_2012} also is a hypothetical policy, for US hospital trusts looking to implement a BYOD policy.
It is short, but interesting as it is written in a different style to other BYOD policies.
The NHS policy~\cite{kennington_mobiles_2014} is a BYOD policy used in a British hospital trust.
It is long, and describes a complex policy in a large organization with a complex hierarchy.
The final two policies are simpler, the fourth is taken from a company selling emergency sirens for cars~\cite{code3pse.org_sample_????}, the fifth is by the University of Edinburgh.  
Both are short, relatively uncomplicated, but typical examples of policies found \emph{in the wild}.

Not all the policies are written in the same style.
Most are written from the perspective of a company or IT department telling an employee what to do.
The HiMSS policy, however, is written as a contract where the user tells the company how they will behave.
Most separate policy rules into small individual rules each which must be followed.
The Edinburgh policy groups them together into two or three large rules, with different increasing sets of sub-rules for low and high risk employees to follow.

To help compare the policies, they were first translated them into a formal language~\cite{hallett_specifying_2016}.
We use a dialect of SecPAL~\cite{becker_secpal:_2010} that we use for describing app privacy preferences~\cite{hallett_apppal_2016}.
SecPAL is designed to be readable, and requires an explicit speaker for all statements.
This allows SecPAL-based languages to capture delegation relationships and the differences in style between the policies.
Both the formalization of the policies\footnote{\url{https://github.com/apppal/apppal-byod-policy-translations}} and tooling for our variant of SecPAL\footnote{\url{https://github.com/apppal/libapppal}} are available online.

\section{Rule Structures and Checks}

Each of the policies are split into a series of rules, which describe the policy.
The SecPAL assertions we used to translate the rules have a common structure:

\begin{center}\ttfamily\footnotesize%
  \newcommand{\sptoken}[1]{$\langle\text{#1}\rangle$}
  \sptoken{speaker} says \sptoken{X} \sptoken{predicate} if \sptoken{Y} \sptoken{predicate}$\cdots$.
\end{center}

The predicates used to describe decisions made by the rules fall into 4 categories.
\emph{Can} predicates describe what their subjects can do; for instance whether a device can connect to a server.
\emph{Must} predicates describe obligations, such as reporting a lost device.
\emph{Has} predicates ensure an action has been completed in the past, such as approving an app.
Finally \emph{is} predicates describe a typing property about their subjects.

The occurrence of each type of predicate in each policy is shown in \autoref{tab:prefix}.
The use of each type is also split by whether the predicate is a \emph{decision} made by the policy, or a \emph{condition} for making that decision.
Can and must-decisions feature in all policies excepting can-decisions in the Edinburgh policy, in part due to the structure of the policy as discussed in \autoref{sec:introduction}.
This is expected, these are access control decisions and reactions to events; both topics that existing MDM tools have focussed on implementing.
Has and is predicates are the majority of the conditions, but there are also decisions using them too.
This suggests that implementing a policy is more complex than choosing what a predefined group of devices can or cannot do.
The groups themselves are defined by policy rules that can dynamically change based on the policy.
Existing MDM tools, such as MaaS360, do not allow for policies to be applied based on the basis of policies.
To implement a BYOD policy fully an MDM tool must be more flexible than current tick-box approaches.

\begin{table}\sffamily\footnotesize\centering
\begin{tabular}{ c  c c c c c c c c c }
\toprule
          & \multicolumn{4}{c}{Decision}              & & \multicolumn{4}{c}{Condition}             \\
Policy    & \rb{Can} & \rb{Must} & \rb{Has} & \rb{Is} & & \rb{Can} & \rb{Must} & \rb{Has} & \rb{Is} \\
\midrule
SANS      & 26       & 22        & 7        & 20      & & 2        & 2         & 8        &  81     \\
HiMSS     & 6        & 12        & 9        & 2       & & 0        & 0         & 3        &  20     \\
NHS       & 13       & 18        & 23       & 16      & & 2        & 0         & 20       &  83     \\
Sirens    & 12       & 20        & 5        & 7       & & 1        & 4         & 1        &  50     \\
Edinburgh & 0        & 2         & 9        & 0       & & 2        & 2         & 15       &  11     \\
\bottomrule \\
\end{tabular}
\caption{Occurrences of predicate-types in each policy.}
\label{tab:prefix}
\end{table}

\section{Common Concerns}

\section{Principals and Delegation}

\section{Conclusions}
\label{sec:conclusions}

Our formalisation of the policies is subjective.

\printbibliography{}
\end{document}
